\documentclass[11pt,a4paper]{article}
\usepackage[hyperref]{acl2017}
\usepackage{times}
\usepackage{latexsym}

\usepackage{url}
\usepackage{booktabs}
\usepackage{graphicx}

%\aclfinalcopy % Uncomment this line for the final submission
%\def\aclpaperid{***} %  Enter the acl Paper ID here

%\setlength\titlebox{5cm}
% You can expand the titlebox if you need extra space
% to show all the authors. Please do not make the titlebox
% smaller than 5cm (the original size); we will check this
% in the camera-ready version and ask you to change it back.

\newcommand\BibTeX{B{\sc ib}\TeX}

\title{Toward a Comparable Corpora of Latvian, Russian and English Tweets}

\author{First Author \\
  Affiliation / Address line 1 \\
  Affiliation / Address line 2 \\
  Affiliation / Address line 3 \\
  {\tt email@domain} \\\And
  Second Author \\
  Affiliation / Address line 1 \\
  Affiliation / Address line 2 \\
  Affiliation / Address line 3 \\
  {\tt email@domain} \\}

\date{}

\begin{document}
\maketitle
\begin{abstract}
\end{abstract}

\section{Introduction}
\label{sec:introduction}

\section{Dataset construction}
\label{sec:construction}

\begin{figure*}[t]
\centering
\includegraphics[width=\textwidth]{supplement/figures/timeline.pdf}
\caption{Tweet counts per day per language. The values are averaged over a week
  window at the right edge.}
\label{fig:timeline}
\end{figure*}

The initial set of tweets was retrieved by subscribing to the \texttt{POST status/filter} endpoint of the Twitter Streaming API\footnote{\url{https://dev.twitter.com/streaming/reference/post/statuses/filter}} with Poultry \cite{dmitrijs_milajevs_2017_546609}. The collected tweets had to be geo-located and had to originate from the area of Riga, the capital of Latvia.\footnotemark{}

\footnotetext{The \texttt{locations} parameter was set to \texttt{23.9325829, 56.8570671, 24.3247299, 57.0859184}}

251\,083 tweets were collected from the period from the 1st of November 2016 to the 31st of March 2017. On April 14th 2017, the collection was rehydrated by querying the Twitter API with the collected tweet IDs to get rid of the deleted tweets. In addition, the tweets that originated retweets were included to the collection: the JSON representation of a retweet includes the original tweet, which we extracted and added to the collection. Rehydrated and expanded collection resulted to 220\,883 tweets. Table~\ref{tab:tweet-counts} summarises the number of collected tweets. 

\begin{table}[h]
  \centering
  \begin{tabular}{lrr}
    \toprule
    Collection & Tweet count \\
    \midrule
    Initial    & 222\,177    \\
    Rehydrated & 220\,883    \\
    Final      & 136\,067    \\
    \bottomrule
  \end{tabular}
  \caption{Tweet counts.}
  \label{tab:tweet-counts}
\end{table}

Further analysis of the extended rehydrated collection showed that there are 23115 (10.5\%) tweets that originated from the user check-ins with Swarm\footnote{\url{https://www.swarmapp.com/}} on Foursquare.\footnote{\url{https://foursquare.com/}} This motivated additional filtering of the rehydrated collection, as ``check-in tweets'' most of the time follow a predefined template and thus do not reflect real language use. 

\begin{table}[h]
  \centering
  \begin{tabular}{lrr}
    \toprule
    Client & Tweet count & Share \% \\
    \midrule
    Twitter Web Client     & 93\,705 &    42.4\% \\
    Twitter for iPhone     & 47\,721 &    21.6\% \\
    Twitter for Android    & 34\,277 &    15.5\% \\
    Foursquare$^*$             & 23\,115 &    10.5\% \\
    Instagram$^*$               & 13\,196 &     5.0\% \\
    Twitter for iPad       & 2\,420  &     1.1\% \\
    Endomondo$^*$               & 1\,611  &     0.7\% \\
    Tweetbot of iOS        & 1\,411  &     0.6\% \\
    World Cities$^*$            & 1\,361  &     0.6\% \\
    Linkis$^*$                  &  660  &     0.3\% \\
    \bottomrule
  \end{tabular}
  \caption{The top ten of Twitter clients in the rehydrated collection. $^*$Clients
    that are not included to the final collection as they do not exhibit
    linguistic value.}
  \label{tab:client-counts}
\end{table}

Table~\ref{tab:client-counts} shows the top ten most popular clients in the rehydrated collection. Together with the tweets originated from Foursquare, tweets from Instagram, an image sharing service), and Endomondo, a workout tracking service, were removed. Tweets written using the ``Word Cities'' client that posts weather reports and the Linkis client, a promotion website, were also removed.

The final collection resulted into 136\,067 tweets which are in Latvian, Russian or English and created after 1st of November 2016. The language of a tweet is provided by the corresponding field in the tweet JSON representation.

\section{Timeline analysis}
\label{sec:timeline}

Out of 136\,067 tweets that constitute the final collection, 45.5\% are in Latvian, 33.9\% are in Russian and 20.7\% are in English, see Table~\ref{tab:language-counts} for tweet counts. The ratio between the Latvian and Russian tweets is roughly the same as the proportion of ethnic Latvians and Russians in Riga, which is 46.2\% to 37.7\%.\footnote{\url{https://en.wikipedia.org/wiki/Riga}}

\begin{table}[h]
  \centering
  \begin{tabular}{lrr}
    \toprule
    Language & Tweet count & Share \% \\
    \midrule
    Latvian     & 61869  & 45.5\%  \\
    Russian     & 46070  & 33.9\%  \\
    English     & 28128  & 20.7\%  \\
    \bottomrule
  \end{tabular}
  \caption{Language distribution in the final collection.}
  \label{tab:language-counts}
\end{table}

Figure~\ref{fig:timeline} shows the bandwidth of tweets in time for all three languages. There are several peaks in Twitter usage. Some of them affect all three languages, as in early January, some of them affect only one language, as in late January.

If the Twitter behaviour is affected by the events in the real world, then the peaks should correspond to events in the real word. The difference in peaks, could then be explained as there are different real word events that trigger discussions on Twitter in Latvian, Russian and English. Table~\ref{tab:timeline-lang-corr} suggests, that tweets in Latvian and English share similar behaviour. The Russian tweet timeline is distinct from both timelines, tough it's behaviour is more similar to the Latvian timeline than to English.

\begin{table}[ht]
  \centering

  \begin{tabular}{lrrr}
\toprule
Language &     Latvian & Russian & English \\
\midrule
Latvian       &  1.0 &  0.4 &  0.6 \\
Russian       &  0.4 &  1.0 &  0.3 \\
English       &  0.6 &  0.3 &  1.0 \\
\bottomrule
\end{tabular}

  
  \caption{Pairwise Pearson's-$\rho$ correlation coefficients between Latvian,
    Russian and English timelines.}
  \label{tab:timeline-lang-corr}
\end{table}

What are the distinctive and similar properties of the timelines? To answer the question, we first identify the real world events that happened during the biggest peaks.

\paragraph{Mid November}

11th of November is \href{https://en.wikipedia.org/wiki/L\%C4\%81\%C4\%8Dpl\%C4\%93sis_Day}{L\={a}\v{c}pl\={e}sis Day}, a memorial day for soldiers who fought for the independence of Latvia. 18 November is the Proclamation Day of the Republic of Latvia. Numerous events take place across the country including candle placing on and by the wall of Riga Castle, torchlight processions and fireworks. The volume of tweets peaks in all three languages, thought the peak in Russian is less significant than in Latvian and English.

% \paragraph{Mid December}

% There is higher amount of Russian tweets, but Latvian and English tweet volume stays constant.

\paragraph{Early January}

There is a lot of activity during the New Year celebration for all three languages. Russian tweets peak the most almost reaching the same number of tweets as tweets in Latvian.

\paragraph{Late January}

\begin{figure*}[t]
\centering
\includegraphics[width=\textwidth]{supplement/figures/score-hist.pdf}
\caption{Histogram of language use uniformity scores. Low values mean that
  distinct languages are used, while high values mean that a single language is preferred.}
\label{fig:score-hist}
\end{figure*}

The inauguration of Donald Trump as the 45th President of the United States held on 20th of January 2017. The number of Latvian tweets increases, while for other languages it stays roughly the same.

% \paragraph{Early March}

% There are peaks in Latvian and Russian tweets.

% \paragraph{Mid March}

% There is a general decline in Twitter usage for Latvian and English, but not for Russian, when for a short period of time there are more Russian tweets posted than Latvian.

Timeline analysis gives an insight of what events are reflected on Twitter, but does not indicate how twitter is used by ethnic communities, as language choice might depend on the topic, rather than author preference.

\section{Language use}
\label{sec:lang-use}

We have seen that the reflection of an event on Twitter depends on the language. How are languages used individually? Do tweet authors tweet in a one language all the time, or switch between languages depending on the context?

We consider 507 users for whom at least 50 tweets were collected. 180 or 35.5\% of them tweet exclusively in one language (75 users tweet only in Latvian, 43 in Russian and 62 in English). Others tweets in several languages.

To get more insight on how languages are used, we compute the language uniformity score defined as:
\begin{equation}
  \label{eq:score}
  \frac{\max(n_\mathit{lv}, n_\mathit{ru}, n_\mathit{en})}{n_\mathit{lv} + n_\mathit{ru} + n_\mathit{en}}
\end{equation}
where $n_\mathit{lv}$ corresponds to the number of tweets in Latvian for a given user, $n_\mathit{ru}$ in Russian and $n_\mathit{en}$ in English.

The higher is the score, the more dominant one language is. The lowest value of 0.33 means that all three languages are used equally. The value of 0.5 means that 50\% of tweets are written in a dominant language.

The histogram on Figure~\ref{fig:score-hist} shows the score distribution. 425 (83.8\%) users tweet mostly in one language (their scores are greater than 0.897). For 78  (15.4\%) users the score is between 0.5 and 0.897. There are only four (0.8\%) users whose dominant language share is less than 50\%.

Among the four Twitter users whose score is less than 0.5---meaning that they use all three languages extensively---three are personal accounts and one is a company account.

Another interesting accounts that tweet equally in Latvian and Russian, but do not tweet in English are the accounts of a library and a football club. It worth noting that mixing languages is not common in Latvian Internet. For example, large media portals tend to have separate Latvian and Russian web-sites and separate Twitter handlers.

To illustrate the language usage pattern between multilingual users, their first most frequently used language, their second most frequently used language and their third frequently languages were identified. Latvian is not only the most used language among the monoligual users, but also is the first and second most common choice between the multilingual users. Russian is the most frequent third choice, refer to Table~\ref{tab:language-use}).

\begin{table}[h]
  \centering
  \begin{tabular}{lrrr}
    \toprule
     & Latvian & Russian & English \\
    \midrule
    Monoligual     & \textbf{75} & 43 & 62  \\
    \addlinespace
    Multi, first   & \textbf{42} & 18 & 22  \\
    Multi, second  & \textbf{58} &  6 & 18  \\
    Multi, third   & 5  & \textbf{55} & 22  \\
    \bottomrule
  \end{tabular}
  \caption{Language choice between monolingual and multilingual users.}
  \label{tab:language-use}
\end{table}

\bibliography{references,dmilajevs_publications}
\bibliographystyle{acl_natbib}

\end{document}
